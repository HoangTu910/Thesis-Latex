\chapter*{Abstract}
\label{Abstract-en}
\addcontentsline{toc}{chapter}{Abstract}

IoT devices often face secure communication challenges due to computational limitations, therefore
increasing their vulnerability of being attacked. To solve these problems, this thesis focuses on 
designing and implementing a complete secure IoT architecture that can be adapted 
to most hardware platforms. The system is built based on the combination of critical techniques: 
the frame-based architecture, dynamic key exchange and session key generation, lightweight cryptography Ascon-128a, 
and the proposed \textit{safe counter} mechanism. The architecture was implemented on STM32F411VET6 and ESP32 hardware, 
along with a Linux server. The results show that the Packet Delivery Ratio (PDR) of the system is 99.99\% over 120 minutes, with
a latency of 51.08 ms. Besides that, when tested by sending 70,000 malicious packets to the server
to simulate a brute-force attack targeting the \textit{sequence number}, the server has successfully rejected 100\% malicious
packets thanks to the \textit{safe counter} mechanism. Also, Ascon-128a has better performance
compared to AES at 78.4\% on the ARM Cortex-M4 processor. The results about memory comsumtion and power consumtion
show that the architecture is suitable to implement on most hardware. The security analysis proves that the system has
the ability to resist some common attacks, claiming the safety of the proposed method. 