\chapter*{Tóm tắt}
\label{Abstract}
\addcontentsline{toc}{chapter}{Tóm tắt}

Các thiết bị IoT thường gặp khó khăn trong việc đảm bảo an toàn truyền thông do hạn chế về tài nguyên tính toán, từ đó làm gia tăng nguy cơ bị tấn công. Nhằm giải quyết vấn đề này, đề tài tập trung phát triển một kiến trúc IoT bảo mật toàn diện, có khả năng triển khai hiệu quả trên nhiều nền tảng phần cứng khác nhau.
Hệ thống được xây dựng dựa trên sự kết hợp của các kỹ thuật cốt lõi như: kiến trúc khung truyền dữ liệu, cơ chế trao đổi khóa động và tạo khóa phiên, thuật toán mã hóa nhẹ Ascon-128a, cùng với cơ chế bảo mật \textit{safe counter} do đề tài đề xuất. Kiến trúc được triển khai thực tế trên nền tảng phần cứng STM32F411VET6, ESP32 và máy chủ chạy hệ điều hành Linux.
Kết quả thực nghiệm cho thấy hệ thống đạt tỷ lệ truyền thành công (Packet Delivery Ratio - PDR) 99.99\% trong 120 phút, với độ trễ trung bình 51.08 ms. Bên cạnh đó, khi tiến hành kiểm thử bằng cách gửi 70,000 gói tin giả mạo nhằm mô phỏng cuộc tấn công brute-force vào \textit{sequence number} của gói tin, hệ thống đã từ chối 100\% số gói tin bất hợp lệ nhờ vào cơ chế \textit{safe counter}. Ngoài ra, thuật toán Ascon-128a cho thấy hiệu suất vượt trội, với tốc độ thực thi nhanh hơn 78.4\% so với phương pháp mã hóa truyền thống AES khi triển khai trên vi xử lý ARM Cortex-M4.
Các kết quả về sử dụng bộ nhớ cũng như mức độ tiêu thụ năng lượng cho thấy kiến trúc có thể phù hợp để triển khai trên đa số nền tảng phần cứng. 
Các phân tích bảo mật trong đề tài cũng chứng minh hệ thống có khả năng chống chịu hiệu quả trước một số dạng tấn công phổ biến, từ đó khẳng định tính an toàn của giải pháp đề xuất.